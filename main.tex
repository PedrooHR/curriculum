\documentclass[11pt,a4paper,sans]{moderncv}
\moderncvstyle{banking} % CV style: 'banking', 'classic', 'casual', 'oldstyle', etc.
\moderncvcolor{blue}    % Theme color: blue, green, red, etc.

% Basic packages
\usepackage[scale=0.75]{geometry}

% Personal Information
\name{\huge Pedro Henrique}{Di Francia Rosso\vspace{0.5em}}
\address{Campinas}{São Paulo}{Brasil}
\email{pedrohrosso@gmail.com}
\social[linkedin]{phrosso}
\social[github]{PedrooHR}

% Document start
\begin{document}

% Header
\makecvtitle

% Professional Summary
\vspace{-1em}
\section{Professional Summary}
Researcher specializing in High-Performance Computing (HPC) with a focus on FPGA acceleration and distributed programming models. Experience in software development, parallel computing, accelerated computing, and fault tolerance for HPC clusters.

% Professional Experience
\section{Experience}
\cventry{2021--Present}{Ph.D. Fellow}{Computer Systems Laboratory, Unicamp}{Campinas, Brazil}{}{
Research on abstractions for FPGA acceleration using OpenMP.\\
- Achieved up to 41\% reduction in programming effort for distributed applications [Publication 1].}
\vspace{.3em}
\cventry{2023}{Research Intern}{Research Labs, Advanced Micro Devices (AMD)}{Dublin, Ireland}{}{
Integrated Ph.D. research with AMD’s FPGA networking tools (ACCL).\\
- Demonstrated up to 41.5\% improvement in bandwidth usage after integrating ACCL [Publication 1].}
\vspace{.3em}
\cventry{2019--2021}{M.Sc. Fellow}{Heuristics Analysis and Learning Laboratory, UFABC}{São Paulo, Brazil}{}{
Developed implementation-independent fault tolerance tools for MPI.\\
- Achieved up to 30\% message reduction during Fault Tolerance broadcasts [Publication 3].}
\vspace{.3em}
\cventry{2017--2018}{Undergraduate Researcher}{Robotics and Automation Laboratory, UFSC}{Araranguá, Brazil}{}{
Developed a wearable electromyography device for outdoor measurements with wireless sensor networks.}

% Education
\section{Education}
\cventry{2021--Present}{Ph.D. in Computer Science}{Unicamp}{Campinas, Brazil}{}{
\textit{Title}: Integrating Multi-FPGA Acceleration to OpenMP Distributed Computing.\\
FAPESP Fellow (Grant No. 2021/09355-2).}

\vspace{.3em}
\cventry{2019--2021}{M.Sc. in Computer Science}{UFABC}{São Paulo, Brazil}{}{
\textit{Title}: OCFTL: An MPI Implementation-Independent Fault Tolerance Library for Task-Based Applications.\\
Finalist for Best Thesis Award (WSCAD-CTD 2021).}

\vspace{.3em}
\cventry{2014--2018}{B.Sc. in Computer Engineering}{UFSC}{Araranguá, Brazil}{}{
\textit{Title}: Elastic Maps Based Image Recoloring for Dichromats.\\
Awarded Best Academic Performance (CREA-SC and UFSC, GPA: 9.04/10).}

\vspace{1em}
% Technical Skills
\section{Technical Skills}
\cvitem{Programming Languages}{C/C++, Python, Shell, Verilog, OpenMP, MPI, CUDA, HLS.}
\cvitem{Tools}{Git, Docker, Linux, Visual Studio Code, LaTeX, SSH.}
\cvitem{Expertise}{Parallel and Distributed Computing, HPC, FPGA Acceleration, Fault Tolerance.}

% Selected Publications
\section{Selected Publications}
\cvitem{[1]}{\textbf{P. H. Rosso}, L. Petrica, N. J. Lisa, et al. \textit{Integrating Multi-FPGA Acceleration to OpenMP Distributed Computing}. IWOMP 2024.}
\cvitem{[2]}{\textbf{H. Yviquel}, M. Pereira, E. Francesquini, et al. \textit{The OpenMP cluster programming model}. ICPP 2022.}
\cvitem{[3]}{\textbf{P. H. Rosso}, E. Francesquini. \textit{OCFTL: An MPI Implementation-Independent Fault Tolerance Library}. CARLA 2021.}
\cvitem{}{For a complete list of publications, please visit my \href{https://scholar.google.com/citations?user=rtONezgAAAAJ}{Google Scholar profile}.}

% Awards and Recognitions
\section{Awards and Recognitions}
\cvitem{2019}{\textbf{Best Academic Performance} (UFSC \& CREA-SC): Top student in Computer Engineering.}
\cvitem{2020/2021}{\textbf{Honorable Mentions} (ERAD-SP): Best graduate papers.}

% Additional Activities
\section{Additional Activities}
\cvitem{Tutorials}{\\ 
- \textbf{2023:} Hands-on workshop on distributed FPGA acceleration using OpenMP (XII INFIERI School, Brazil).\\
- \textbf{2024:} Introduction to distributed FPGA acceleration (XVI Santos Dumont Summer School, remote).}

% End of document
\end{document}
